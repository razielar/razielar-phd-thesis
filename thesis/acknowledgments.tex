I would like to express my profound gratitude to my thesis co-directors, Roderic Guigó and Montserrat Corominas, for their feedback, guidance, and for providing a research environment that was intellectually stimulating and in which I was able to improve my research skills. My work was also co-supervised by Assaf Bester, to whom I am also grateful for his guidance, and for the feedback in the second part of this study.  

Besides Roderic, Montse and Assaf, I also thank Carlos Camilleri, for his remarkable scientific contributions to the improvement of this thesis. I am grateful to all members of Roderic’s lab.
The administrative support provided by Montse Ruano, and Romina Garrido have been exceptional. This work also benefited from the computational help of Emilio Palumbo. 

Many thanks to the following people for useful feedback on various parts of the Thesis manuscript: Montserrat Corominas, Manuel Muñoz, David Brena, Reza Sodaei, Marc Elosua, Iman Sadeghi, and Roderic Guigó. 

Additionally, the following non-exhaustive list of people, which positively impacted my PhD experience and this work:  

Cecilia C. Klein, for your mentorship during the first year of PhD and share your insights in \textit{Drosophila}, regeneration and bioinformatic. From you I learned how to combine biology and computational analysis to obtain novel results. 

Manuel Muñoz, for making my adaptation process in the lab easier; help in statistics, plotting, programming, coding best practices; for introducing me into the Emacs world (the most efficient text editor/IDE) and, last but not least, the fun hackathon times we shared, we learned a lot. Your examples helped me realize the importance of hard working, curiosity-driven projects and never stop learning.

Reza Sodaei and Valentin Wucher, for giving the opportunity to collaborate with you in the circadian-seasonal manuscript. I learned why the science-core should be based on collaboration and use different expertise to solve a problem, and achieve a more complete conclusion.  

Iman Sadeghi, for your friendship, all the adventures we had through our PhD years, personal advice, and time to time scientific counseling in this research. Julien Lagarde, it was very helpful in sharing his LaTeX code to everybody,  using open-source software, providing thorough instructions of how his code works and helping when I needed it. Vasilis Ntasis, for the geek talks, sharing your linux/R tools and philosophy of mostly using the keyboard made me more productive and efficient.    

Thanks to my love Vanessa Vega for her constant support, in graphic design, revisions, polishing plots, and otherwise.

Por último, agradezco a mis padres y hermanos por todo el apoyo, amor y proporcionarme las herramientas necesarias para ser quien soy. Sin ustedes esto no sería posible.  


