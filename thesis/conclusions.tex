The main conclusion of the present Thesis Project are the following: 

\begin{enumerate}

\item Adding cell-specific ENCODE TF ChIP-seq data to CRISPRi functional screen data improves ML model performance and increases the biological explainability for hit predictions. Moreover, for our hit dataset, acting on the cost-function instead of under-sampling the majority class (not hit) shows better performance for AUROC, sensitivity, and specificity metrics.

\item Cost-sensitive XGBoost classifier with 71 features (16 genomic features plus 55 TF ChIP-seq related features) is 10\% more reliable, in terms of AUROC, than other algorithms in discerning between hits and not hits. Additionally, sensitivity and specificity values are balanced across the seven human cell lines. 
  
\item Hit predictions from our trained classifier are a valuable tool to uncover lncRNAs affecting cell-growth rates. The lncRNA \textit{LINC00879} is a successful example for our ML algorithm. Further, "\textit{Distance between lncRNA-TSS and PC}", "\textit{expression level}", and "\textit{number of TFs with ChIP-seq signal}" are the top 3 most important features for our classifier.

\item There are key lncRNAs involved during \textit{Drosophila} wing imaginal disc regeneration process. Such lncRNAs are mainly present at the early stage with low sequence-conservation; presenting time point and condition specific expression patterns.

\item Upon \textit{CR40469} genetic deletion in regeneration conditions, there is a significant transcriptomic alteration. Such differentially expressed genes are mostly localized in the X chromosome, suggesting a \textit{trans-acting} mechanism of the lncRNA \textit{CR40469} in the fruit fly X chromosome.  

\end{enumerate}


